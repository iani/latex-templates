\section{Τίτλος Μαθήματος}
\label{sec:org5dbd0d0}

Music, Tradition, Avant-Garde. Music as Source of Knowledge

Ελληνικός Τίτλος: Μουσική, Παράδοση, Πρωτοπορία.  Μουσική σαν Πηγή Γνώσης.
\subsection{Σημείωση για τον Υπότιτλο του Μαθήματος}
\label{sec:org42e7b88}

Ο υπότιτλος "Μουσική σαν Πηγή Γνώσης" επιλέχθηκε επειδή η διαχρονική αναφορά σε εποχές πρίν της μοντέρνας δυσκολεύει ή κάνει ανάρμοστη την χρήση όρων ή διακρίσεων που επεκράτησαν την μοντέρνα εποχή όπως: Τεχνολογία, Επιστήμη, Μέσα, Πολιτισμός, Τελετουργία, Μυθολογία, Ψυχολογία, Ανθρωπολογία κοκ.  Έτσι γίνεται αδύνατο ή άκομψο να διατυπωθεί ο τίτλος με ένα σκέπτικο σαν "Η Μουσική στην Τομή ανάμεσα σε Τεχνολογία, Επιστήμη, Μέσα, Πολιτισμό, Τελετουργία \ldots{} " (κλπ.).  Η έννοια της Γνώσης καθαυτή αρμόζει πολύ καλύτερα στην διαχρονική και αποστολή της μουσικής ανα τους πολιτισμούς.
\subsection{Σκοπός του Μαθήματος}
\label{sec:org1f16afd}

Σκοπός του Μαθήματος είναι η να διασυνθέσει μουσικά έργα και φαινόμενα με τις ανθρώπινες αναζητήσεις και ιστορικές συγκυρίες των πολιτισμών στα οποία αυτά δημιουργήθηκαν. Παράλληλα το μάθημα αποκαλύπτει και τονίζει πρωτοποριακά στοιχεία σε έργα ανεξάρτητα από την εποχή που δημιουργήθηκαν, και οδηγεί κατ'αυτόν τον τρόπο σε μια ευρύτερη θεώρηση της έννοιας της πρωτοπορίας. 

\subsection{Θεματολογία και δομή}
\label{sec:org30f4830}

\section{Πλανο διαλέξεων}
\label{sec:orgeac4540}
\subsection{Μουσική και Κοσμολογία}
\label{sec:org90d22e4}
\subsubsection{Πυθαγόρας}
\label{sec:org35ad389}
\subsubsection{Κλαύδιος Πτολεμαίος}
\label{sec:orgaa99d0a}
\subsubsection{Κονφούκιος και Guchin}
\label{sec:org1f21a7a}
\subsubsection{Νεοπλατωνισμός και Δερβίσηδες Μεβλεβί}
\label{sec:org3531576}
\subsubsection{Κέπλερ}
\label{sec:orgaebcc4c}
\subsubsection{Νεύτωνας}
\label{sec:org9fa94d4}
\subsubsection{J.S. Bach}
\label{sec:org01660da}
\subsubsection{Jean Rebel: Les Elements}
\label{sec:orgadfa142}
\subsubsection{J. Haydn: Die Shöpfung}
\label{sec:orgaefdbb4}
\subsubsection{Poincare}
\label{sec:org7063928}
\subsubsection{Ξενάκης: La légende d'éer}
\label{sec:org52e6a22}
\subsection{Μουσική και Πάθος}
\label{sec:orgd72e2fa}
\subsubsection{Dowland: Flow my Tears. Lacrimae}
\label{sec:orgdb39dc5}
\subsubsection{Monteverdi: Lamento d'Arianna. Lamento della ninfa, Μαδριγάλια σε κείμενα των Petrarca και Torquato Tasso.}
\label{sec:orgebbd75b}
\subsubsection{Gesualdo}
\label{sec:orgd70e33b}
\subsubsection{J.S. Bach: Πάθη κατά Ματθαίον: Μερος 2, Αρια πρώτη, Λειτουργία σε Σι ελάσσονα: Agnus Dei, Πάθη κατα Ιωάννην Σκηνές από την Σταύρωση, Chaconne for Solo Violin.}
\label{sec:orgbb02574}
\subsubsection{Ξενάκη Ορέστεια}
\label{sec:orgf6a8ad6}
\subsection{Μουσική και Πολιτική / Συλλογικότητα}
\label{sec:org4c18249}
\subsubsection{Vivaldi: Justinus}
\label{sec:org70ae522}
\subsubsection{Mozart The Marriage of Figaro, La Clemenza di Tito}
\label{sec:org6dbffd0}
\subsubsection{Puccini Tosca}
\label{sec:org422bb04}
\subsubsection{}
\label{sec:orge58864b}
\subsubsection{Nono an Diotima, Opera Rehearsal}
\label{sec:org5245122}
\subsubsection{Μεταστάσεις}
\label{sec:orgcd64d6d}
\subsection{Μουσική και Πόλεμος}
\label{sec:orgafb8454}
\subsection{Μουσική και Χορός. Έκσταση και Μεταμόρφωση}
\label{sec:org6967fef}
\subsubsection{Εκστατικοί χοροί, Τελετουργικές Παραστάσεις}
\label{sec:org7084a76}
\begin{enumerate}
\item Ketchak
\label{sec:org86a553f}
\item Noh
\label{sec:orgb6c2b0c}
\item Dhikr
\label{sec:org9101f5e}
\item Sema
\label{sec:org25396b9}
\item Gigue (?)
\label{sec:org092fdc7}
\item Tarantella και Ζωναράδικος
\label{sec:orgc0884f3}
\item Waltz
\label{sec:org387ee6d}
\item Zarzuela, Zorcico, Bolero
\label{sec:org9b14812}
\end{enumerate}
\subsubsection{Χρήστου: Μεταστάσεις, Πύρινες Γλώσσες, Ορέστεια}
\label{sec:orgda1e9cc}
\subsubsection{Μουσική και ταυτότητα /}
\label{sec:org5feb74b}
\subsubsection{Μεταμορφωτικοί Χοροί}
\label{sec:orgf69106f}
\subsubsection{Μεταμόρφωση στο μουσικό Θέατρο}
\label{sec:org5c8c474}

\subsection{Μουσική και Σώμα 1: Βάδισμα}
\label{sec:org14a9079}
\subsection{Μουσική και Σώμα 2: Αναπνοή, Χτύπος Καρδιάς}
\label{sec:org1a2379f}
\subsection{Αναίρεση του Χρόνου: Ετεροχρονισμός, Επιστρωμάτωση, Κυκλικές Φόρμες}
\label{sec:org4528e11}
\subsubsection{J. S. Bach: Mass in b minor: Confiteor - Et resurrexit.}
\label{sec:org50203a1}
\subsubsection{Schubert: Quartet No 5 in G major, 1st movement}
\label{sec:orgfd867c6}
\subsubsection{Chopin: Scherzo in b minor}
\label{sec:org0b0f7d9}
\subsection{Μουσική και το Φανταστικό}
\label{sec:org7867434}
\subsubsection{J. Haydn: L'isola desabitata}
\label{sec:org783a300}
\subsubsection{L. v. Beethoven: Harfenquartett}
\label{sec:orgdfe669e}

\subsubsection{Chopin: Nocturnes,}
\label{sec:org173b6d5}
\subsection{Μουσική και Ψυχανάλυση / Το Ονειρικό και το Μεταβατικό}
\label{sec:orgda7dae6}
\subsubsection{}
\label{sec:org4b7145f}
\subsection{Μουσική, Αφήγηση, Γλώσσα, Γραφή}
\label{sec:org01d21d0}
\subsection{Μουσική και Εθνολογία / Έθνη}
\label{sec:org8786772}
\subsubsection{Η γένεση της έννοιας του Έθνους από μουσική πλευρά: Schubarth και Schubert}
\label{sec:orgdf62755}
\subsubsection{Λαογραφικά στοιχεία πριν τις εθνικές σχολές: Vivaldi, Bach, Haydn, Mozart, Beethoven, Schubert, Chopin, Lisz.}
\label{sec:orgc1ac624}
\subsubsection{Chopin: Mazurkas}
\label{sec:org3c82ce0}
\subsubsection{Bartok σαν πατέρας της Εθνομουσικολογίας}
\label{sec:org4243220}
\subsubsection{Εξωτισμός, Οριενταλισμός και Αποικιοκρατία: Verdi Aida, Havely la Juive, Salammbô (Mussorgsky, Rachmaninoff, Reyer, Hauer, Fénelon), Debussy's pentatonism, Puccini Madame Butterfly, La fanciulla del West, Turandot}
\label{sec:org6b0bf01}
\subsection{Μουσική, Θέατρο και Δυνητικοί Κόσμοι}
\label{sec:orgf305c7c}
\subsection{Ουτοπικά Μουσικά Πρότζεκτ και η γένεση της σύγχρονης Πρωτοπορίας}
\label{sec:orgf9f71ec}
\subsubsection{}
\label{sec:org43c8713}
